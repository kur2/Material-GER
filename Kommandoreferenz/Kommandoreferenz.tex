\documentclass[a4paper,10pt]{article}

\usepackage{ngerman}
\usepackage[utf8]{inputenc}
\usepackage{amsmath}
\usepackage{amssymb}
\usepackage{multirow}


\usepackage[left=2.2 cm,right=2.2 cm,top=1.5cm,bottom=2.0cm]{geometry}

\newcounter{aufgnr}

\begin{document}
\begin{center}
\section*{KUR2}
\subsection*{Kommandoreferenz}
\end{center}

\hrule
\vspace{.5cm}

\begin{table}[h]
\begin{tabular}{l|l|c|l|p{6.85cm}}
\multicolumn{1}{c|}{\textbf{Domäne}} & \multicolumn{1}{c|}{\textbf{Bezeichnung}} & \multicolumn{1}{c|}{\textbf{Code}} & \multicolumn{1}{c|}{\textbf{Argument}} & \multicolumn{1}{c}{\textbf{Beschreibung}} \\
\hline
0 - Fluss & NOOP & 00 & --- & Führt keine Operation aus\\
& HALT & 01 & --- & Haltebefehl: Signalisiert das Programmende und hält die Maschine an\\
\hline
1 - Sprünge & JUMP & 10 & ZIEL & Springt zur Speicheradresse ZIEL\\
& JUMP=0 & 11 & ZIEL & Springt zur Speicheradresse ZIEL, falls der Akkumulator 0 enthält\\
& JUMP$\neq$0 & 12 & ZIEL & Springt zur Speicheradresse ZIEL, falls der Akkumulator nicht 0 enthält\\
& JUMP$>$0 & 13 & ZIEL & Springt zur Speicheradresse ZIEL, falls der Akkumulator einen positiven Wert enthält\\
& JUMP$<$0 & 14 & ZIEL & Springt zur Speicheradresse ZIEL, falls der Akkumulator einen negativen Wert enthält\\
\hline
2 - Daten & NULL & 20 & --- & Lädt den Wert 0 in den Akkumulator\\
& LOADC & 21 & KONSTANTE & Lädt den Wert KONSTANTE in den Akkumulator\\
& LOADA & 22 & ADRESSE & Lädt den Wert aus der Speicheradresse ADRESSE in den Akkumulator\\
& STORE & 23 & ADRESSE & Schreibt den Wert aus dem Akkumulator in die Speicheradresse ADRESSE\\
\hline
3 - Addition & ADDC & 30 & KONSTANTE & Addiert zum Wert im Akkumulator den Wert KONSTANTE\\
& ADDA & 31 & ADRESSE & Addiert zum Wert im Akkumulator den Wert aus der Speicheradresse ADRESSE\\
\hline
4 - Subtraktion & SUBC & 40 & KONSTANTE & Subtrahiert vom Wert im Akkumulator den Wert KONSTANTE\\
& SUBA & 41 & ADRESSE & Subtrahiert vom Wert im Akkumulator den Wert aus der Speicheradresse ADRESSE\\
\hline
5 - Multiplikation & MULC & 50 & KONSTANTE & Multipliziert den Wert im Akkumulator mit dem Wert KONSTANTE\\
& MULA & 51 & ADRESSE & Multipliziert den Wert im Akkumulator mit dem Wert aus der Speicheradresse ADRESSE\\
\hline
6 - Division & DIVC & 60 & KONSTANTE & Dividiert den Wert im Akkumulator durch den Wert KONSTANTE\\
& DIVA & 61 & ADRESSE & Dividiert den Wert im Akkumulator durch den Wert aus der Speicheradresse ADRESSE\\
\hline
7 - Modulo & MODC & 70 & KONSTANTE & Bildet den Rest der Division des Wertes im Akkumulator durch den Wert KONSTANTE\\
& MODA & 71 & ADRESSE & Bildet den Rest der Division des Wertes im Akkumulator durch den Wert aus der Speicheradresse ADRESSE\\
\hline
8 - I/O & INP & 80 & QUELLE & Lädt den nächsten Wert des Eingabegerätes, das an den Anschluss QUELLE angeschlossen ist in den Akkumulator\\
& OUT & 81 & ZIEL & Gibt den Wert im Akkumulator an das Ausgabegerät, das an den Anschluss ZIEL angeschlossen ist in den Akkumulator\\

\end{tabular}
\end{table}

\end{document}