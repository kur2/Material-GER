\documentclass[a4paper,12pt]{article}

\usepackage{ngerman}
\usepackage[utf8]{inputenc}
\usepackage{amsmath}
\usepackage{amssymb}
\usepackage{multicol}
\usepackage{framed}
\usepackage{graphicx}
\usepackage{hyperref}

\usepackage[left=2.2 cm,right=2.2 cm,top=1.5cm,bottom=2.0cm]{geometry}

\newcounter{aufgnr}

\begin{document}
\begin{center}
\section*{Die Von-Neumann-Architektur}
\subsection*{Beschreibung des Modelles}
\end{center}

\hrule
\vspace{.5cm}

Das Von-Neumann-Modell (auch: Von-Neumann-Architektur) ist eine Beschreibung einer universellen Rechenmaschine. Es beschreibt also eine Maschine, die nicht eine feste Aufgabe erfüllt, sondern prinzipiell jedes berechenbare Problem lösen kann. Entwickelt wurde es 1945 vom ungarisch-amerikanischen Informatiker John von Neumann und findet sich heute in den meisten bekannten Rechenmaschinen (Computern).\\

\subsubsection*{Prinzipien des Modelles:}
\begin{itemize}
\item Die Maschine ist unabhängig vom Problem (bzw. Zweck)
\item Der Speicher ist in
	\begin{itemize}
	\item aufeinander folgende,
	\item linear adressierte (durchnummerierte)
	\item gleich große
	\end{itemize}
	Speicherzellen aufgeteilt
\item Befehle (Programme) und Daten befinden sich im selben Speicher
\item Programme sind in aufeinander folgenden Speicherzellen abgelegt und werden ausgeführt, indem die Befehle in der Reihenfolge ausgeführt werden, in der sie im Speicher stehen
\item Von dieser Reihenfolge kann durch (evtl. bedingte) Sprungbefehle abgewichen werden
\end{itemize}

\subsubsection*{Komponenten eines Von-Neumann-Rechners}
\begin{itemize}
\item Rechenwerk
\item Steuerwerk
\item Speicherwerk
\item Eingabewerk
\item Ausgabewerk
\end{itemize}
Das Rechen- und Steuerwerk werden häufig zusammengefasst und als CPU (\emph{Central processing unit}, engl. für zentrale Verarbeitungseinheit) bezeichnet.


\end{document}